\chapter{前言}

这本书是模仿 \href{https://r4ds.had.co.nz/}{\it{R for Data Science}} 一书的结构进行编排的。这本书的编写旨在帮助你学习如何使用 Stata 进行数据整理和建模。虽然这本书的名字是 \it{Stata for Data Science},但是这本书的重点是如何使用 Stata 进行数据整理、分析和简单建模。这也是 \href{https://r4ds.had.co.nz/}{\it{R for Data Science}} 一书的主要内容。本书还提供了网页阅读:\href{https://www.czxa.top/stata4ds/book/}{Stata for Data Science}。在阅读中,如果您遇到了任何问题,欢迎在 \href{https://github.com/czxa/stata4ds}{GitHub} 上给我提交 \href{https://github.com/czxa/stata4ds/issues}{issues} 或者邮件 \footnote{\email{czxjnu@163.com}.}联系我。

\section{从 Stata 到 R 再到 Stata}

我是在学习  \href{https://r4ds.had.co.nz/}{\it{R for Data Science}}  的时候萌生的写这本书的想法。而我的编程语言的学习顺序是先学 Stata 后学 R。所以编写这本书实际上是在学习 R 语言的时候复习 Stata,所以我称之为:\textcolor{third3}{Stata $\rightarrow$ R  $\rightarrow$ Stata}。

我的统计软件学习顺序是 Stata 到 R的。我第一次接触 \href{https://www.stata.com/}{Stata} 是在 2016年我上大二的时候,那个学期我去蹭了\href{https://ec.jnu.edu.cn/news/view/id/4156}{张宁老师}的计量经济学。Stata 并非我学习的第一门编程语言(我的第一门编程语言应该是Java,不过我并没有继续学下去),但却是我第一门认真学习的编程语言,或者更具体地说,第一门统计编程语言。在随后的两年时间里,我先后学习了 Stata 在计量经济学中的使用、Stata 数据处理、Stata 网页数据爬取以及Stata 图表绘制。除此之外,还学习了一些 Mata 方面的东西,尽管在 Stata 方面花费了如此之大的功夫,我依然感觉自己对 Stata 的掌握不够系统。因此写这本书有四个目的:

\begin{enumerate}
 \item 学习 \href{https://bookdown.org/}{bookdown} 包的使用,这个包可以非常方便的用于书籍排版;
 \item 整理过去两年的 Stata 笔记;
 \item 复习 \href{https://r4ds.had.co.nz/}{R for Data Science} 一书。
 \item 帮助我的好朋友们学习 Stata。
\end{enumerate}

我的 Stata学习大致到大三下学期就结束了,之后我又努力地学习了一段时间的 R 和 Python。在比较熟练的掌握了 R 数据分析技能之后,我便很少再使用 Stata 了,但是我依然觉得 Stata 是一门非常优秀且强大的编程语言。在 Stata 的诸多优点之中,我尤其喜欢 Stata 的帮助文档,非常详细。

我用 Stata 完成了我大学除了毕业论文之外的所有论文,这些使用经验告诉我,Stata 是一门可以信赖的统计软件。

\begin{note}
本书正在编写中,因此书中有大量纰漏,敬请雅正。
\end{note}

\section{本书内容}

由于本书是按照 \it{R for Data Science} 一书的结构组织的,因此本书的结构与之类似:

\begin{enumerate}
  \item 探索。主要是介绍 Stata 的基本操作,比如如何安装和更新 Stata、如何导入 Stata 的系统数据集并进行简要的整理分析和画图。

  \item 深入。本部分将会更加深入地介绍使用 Stata 处理数据的一些技巧。包括日期、字符串、数值变量的处理、数据长宽转换等。

  \item 编程。由于作者对 mata 的了解几乎没有,所以这一部分的编程当然是指 ado 的编程,通过学习 ado 编程,你可以创建自己的 Stata 命令。在这部分还会介绍 Stata 中的 local 和 global 变量以及循环的使用。

  \item 模型。由于本书的重点不在于计量经济学,因此这一部分仅以最简单的 OLS 模型为例介绍。

  \item 汇报。Stata15 引入了一些新东西,例如 **putdocx**, 这个命令可以让你直接使用 Stata 创建 Word 文档。这可能有些类似于 RMarkdown。除此之外,我还会介绍一些用于在 Stata 工作流程中使用 Markdown 的外部命令。
\end{enumerate}

\section{阅读本书之前的准备工作}

首先你需要在你的 Windows 上或者 Mac 上安装 Stata15,由于作者的电脑是 Mac 系统,所以本书的内容尚未在 Windows 上测试。如果你运行出错,请联系作者。

另外,我再向你推荐一个非常好用的代码编辑器:\href{http://www.sublimetext.com/}{Sublime Text 3},Stata 的安装和 Sublime Text3 的配置教程网上有很多,作者的个人网站上也有一些:\href{https://www.czxa.top/posts/59313/}{Stata安装与Sublime Text3配置教程},因此这里不再赘述。

为了方便本书的阅读,作者编写了一个 Stata 的命令包,你可以运行下面的命令安装:

\begin{lstlisting}
  * 首先需要安装 github 命令,这个命令可以用来安装 GitHub 上的 Stata 命令。
  net install github, from("https://haghish.github.io/github/")
  * 然后使用 github 命令安装 stata4ds
  github install czxa/stata4ds, replace
\end{lstlisting}

这个命令包会随着本书的更新而更新。因此在学习本书前,请确保先更新 stata4ds 命令包。

\section{目标读者}

本书的目标读者是经济和金融专业的本科同学(毕竟作者也只是个渣渣本科)。读者并不需要有 Stata 的基础,但是建议读者对 Stata 的使用有一定的了解(例如知道在哪写代码,怎么运行)。

\section{排版约定}

在本书中,你会发现一些不同的文本样式,用以区别不同种类的信息,这里举例说明一些样式,以及它们的含义:

代码的输入和输出格式如下:

\begin{lstlisting}
  * 导入系统数据集
  clear all
  sysuse auto, clear
  *> (1978 Automobile Data)
\end{lstlisting}

\textcolor{second1}{$*$} 开头的行为注释。\textcolor{second1}{$*>$} 开头的行为运行结果。\textbf{新术语} 和 \textbf{重要的词} 用黑体表示。

\section{下载示例代码}

本书的代码开源在 GitHub 上,你可以从这里下载:\href{https://github.com/czxa/stata4ds}{stata4ds}。

\section{许可证}

本书是一本开源书籍,使用 \href{https://creativecommons.org/licenses/by-nc-nd/3.0/us/deed.zh}{Creative Commons Attribution-NonCommercial-NoDerivs 3.0} 许可证。这意味着:

\begin{figure}[htbp]
  \centering
  \includegraphics[width = \textwidth]{assets/license.png}
  \caption{Creative Commons Attribution-NonCommercial-NoDerivs 3.0}
  \label{fig:license}
\end{figure}

如果你想支持作者的工作,欢迎前往 \href{https://www.czxa.top}{作者的网站}对作者进行打赏。你的支持将会促使作者更加及时地更新这本书。

\section{读者反馈}

欢迎读者的反馈。你对本书有任何想法,喜欢或者不喜欢什么,请告知我。你可以在下面的评论区里评论,如果你阅读的是 PDF 版本,你可以前往 \href{https://wwww.czxa.top/stata4ds}{Stata for Data Science} 创建 \href{https://github.com/czxa/stata4ds/issues}{issues}。
